\addchap*{Notation}
\markboth{Notation}{}

\addsec*{Allgemeine Bezeichner}

\begin{tabular}{p{1.5cm} p{12cm}}
$\alpha$, \dots, $\omega$ & Skalare \\
$\veca$, \dots, $\vecz$ & Skalar, Vektor, Funktionssymbol (oder Realisierung einer Zufallsvariablen) \\
$\rdveca$, \dots, $\rdvecz$ & Zufallsvariable (skalar bzw. vektoriell) \\
$\rdesveca$, \dots, $\rdesvecz$ & Schätzer für jeweilige Variable als Zufallsgröße \\
$\esveca$, \dots, $\esvecz$ & Realisierter Schätzer für jeweilige Variable \\
$\mA$, \dots, $\mZ$ & Matrix \\
$\rdmA$, \dots, $\rdmZ$ & Matrix als Zufallsgröße\\
$\sA$, \dots, $\sZ$ & Menge \\
$\ssA$, \dots, $\ssZ$ & Mengensystem \\
\end{tabular}

\addsec*{Spezielle Bezeichner}

\begin{tabular}{p{1.5cm} p{12cm}}
$t$ & Spezielle Bezeichner mit konkreter Bedeutung in dieser Arbeit, z.\,B. $t$ Zeitindex\\
\end{tabular}

\addsec*{Allgemeine Mengen}

\begin{tabular}{p{1.5cm} p{12cm}}
$\dsC$ & Menge der komplexen Zahlen \\
$\dsH$ & Poincaré Halbebene\\
$\dsN$ & Menge der natürliche Zahlen (ohne Null) \\
$\dsN_0$ & Menge der natürliche Zahlen mit Null \\
$\dsQ$ & Menge der rationalen Zahlen \\
$\dsQ^{>0}$, $\dsQ^{<0}$ & Menge der positiven bzw. negative rationalen Zahlen \\
$\dsR$ & Menge der reellen Zahlen \\
$\dsR^{>0}$, $\dsR^{<0}$ & Menge der positiven bzw. negative reellen Zahlen \\
$\dsZ$ & Menge der ganzen Zahlen \\
\end{tabular}

\addsec*{Spezielle Symbole}

\begin{tabular}{p{1.5cm} p{12cm}}
$\normDist( \mu, \sigma^2 )$ & Normalverteilung mit Erwartungswert $\mu$ und Varianz $\sigma$ \\
$\fisherDist_{r,s}$ & Fisher-Verteilung mir $r$ Zähler- und $s$ Nennerfreiheitsgraden\\
$\studentDist_s$ & Student-$t$-Verteilung mit $s$ Freiheitsgraden \\
$\diracDist_\xi$ & Ein-Punkt-Maß an der Stelle $\xi$ \\
$\chiSqDist_s$ & $\chiSqDist$-Verteilung mit $s$ Freiheitsgraden \\
\end{tabular}
